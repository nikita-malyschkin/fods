\documentclass[a4paper,10pt]{article}
\usepackage{geometry}
\usepackage{mathtools}
\usepackage{amsfonts}
\usepackage{amsmath}
\usepackage{forest}

\usepackage[utf8]{inputenc}

 \geometry{
 a4paper,
 total={170mm,257mm},
 left=20mm,
 top=30mm,
 }
\usepackage{fancyhdr}
\fancyhead[L]{Leonard Korp 302582 \\ Nikita Malyschkin 319500 \\ Valentin Engelke 358096}
\fancyhead[R]{WS 2017/18 \\ 12.10.2017}
\pagestyle{fancy}



\begin{document}
\centerline{\Large\bfseries  Foundations of Data Science }
\centerline{\bfseries  Exercise sheet 1}

\section*{Exercise 1}
 \subsection*{a)}
 $f(x,y)=2x+y$
  \subsection*{b)}
  $f(x,y)=ax+by+c$ \\
  (-2,0) is classified as +1, (-1,0) as -1, therefore $f(-2,0) > f(-1,0)$ and $a<0$. \\
    (1,0) is classified as -1, (2,0) as +1, therefore $f(2,0) > f(1,0)$ and $a>0$. \\ This is contradiction, so there exists no linear function that correctly classifies $S_2$.
   \subsection*{c)}
The function $f(x,y)= 2x^2 +y-4$ correctly classifies every point in $S_2$. \\
Therefore $S_2$ is realisable in $\mathcal{H}_1$
\section*{Exercise 2}
$\begin{tabular}{|c|c|c|c|c|}
\hline
Point & euclidean,k=2 & euclidean,k=3 & manhattan,k=2 & manhattan,k=3 \\
\hline
(4,3,3) & 1&1&1&1 \\
(4,-1,-1) & 1&-1&1&1 \\
(-2,4,5) & -1&-1&1&1 \\
(-2,-6,-1)& 1 & 1&1&1 \\
(6,0,2) & 1&1&-1&-1 \\
\hline
\end{tabular}$ \\
Note: points, where there is a tie have been classified as '1'
\section*{Exercise 3}
\begin{forest}
for tree={circle,draw}
[z,
	[1,edge label={node[midway,left] {1}}]
 	[x,edge label={node[midway,right] {0}},
  		[z,edge label={node[midway,left] {1}},
			[1,edge label={node[midway,left] {1}}]
			[0,edge label={node[midway,right] {0}}]  		
  		]
  		[0,edge label={node[midway,right] {0}}]
 	] 
] 
\end{forest} \\
Reasoning for Splits:\\
1. z was chosen since it appears twice in the function, so if it was not chosen first, two 'z' nodes would be needed. \\
2. now either x or z can be chosen, it does not make a difference which is chosen first
\section*{Exercise 4}
\subsection*{a)} 
$P(1,2,3),Q(4,5,6) \\ \\
h_1=  \begin{pmatrix}-2\\1\\1 \end{pmatrix} \cdot x=3 \\\\
h_2=  \begin{pmatrix}0\\1\\-1 \end{pmatrix} \cdot x=-1 \\
$
The intersection of the two planes is 1-dimensional (a line). \\
g=$   \begin{pmatrix}-2\\1\\0 \end{pmatrix} + \begin{pmatrix}1\\1\\1 \end{pmatrix}  \cdot t \\$ \\
Es ist nicht möglich die Gerade in der Form $a\cdot x =b$ zu schreiben.
\subsection*{b)}
Both planes are not homogeneous. \\
$h_{1'}=  \begin{pmatrix}-2\\1\\1 \end{pmatrix} \cdot x = 0 \\
h_{2'}=  \begin{pmatrix}0\\1\\-1 \end{pmatrix} \cdot x = 0 \\ $
\subsection*{c)}
There are an infite amount of hyperplanes containing both P and Q since we can rotate the plane around the axis between P and Q. It is not possible to choose a Point R, so it would be impossible to construct a hyperplane containing P,Q,R as mentioned above, the plane can be freely rotated to contain every possible Point R.  
\subsection*{d)}
Choose $R'$ so it lies on the line that P and Q form and $S'$ so it does not lie on the line. \\
For example: $ R' = \begin{pmatrix}0\\1\\2 \end{pmatrix}, S' = \begin{pmatrix}0\\0\\0 \end{pmatrix}$ \\
Three non-collinear points (e.g. P,Q,S') define exactly one plane.
\end{document}
