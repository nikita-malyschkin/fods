\documentclass[a4paper,10pt]{article}
\usepackage{geometry}
\usepackage{mathtools}
\usepackage{amsfonts}
\usepackage{amsmath}
\usepackage{listings}

\usepackage[utf8]{inputenc}

 \geometry{
 a4paper,
 total={170mm,257mm},
 left=20mm,
 top=30mm,
 }
\usepackage{fancyhdr}
\fancyhead[L]{Leonard Korp 302582 \\ Nikita Malyschkin 319500 \\ Valentin Engelke 358096}
\fancyhead[R]{WS 2017/18 \\  14.12.2017}
\pagestyle{fancy}



\begin{document}
\centerline{\Large\bfseries  Foundations of Data Science }
\centerline{\bfseries  Exercise sheet 9}
\section*{Exercise 3}
Let $n_1$ be number of entries with the value 1 in the matrix. Then the space needed for the first approach is $n*n*1$ bit . The second approach needs $n_1 * \lceil log_2 n \rceil$ bit. 
\[n_1 * \lceil log_2 n \rceil < n*n \Leftrightarrow \frac{n_1}{n*n} < \frac{1}{\lceil log_2 n \rceil} \]
This means, that the fraction of 1s has to be lower than $\frac{1}{\lceil log_2 n \rceil}$ for the second approach to save space.
\section*{Exercise 4}
\subsection*{a)}
\begin{lstlisting}
function MAP(value)
	max=INTEGER.MIN_VALUE
	for all v in value do
		if(v>max)
			max=v
	emit(1,max)
	
function REDUCE(key,values)
	max=INTEGER.MIN_VALUE
	for all v in values do
		if(v>max)
			max=v
	emit(max)
\end{lstlisting}
\subsection*{b)}
\begin{lstlisting}
function MAP(value)
	for all v in value do
		emit(1,v)

	
function REDUCE(key,values)
	sum=0
	n=0
	for all v in values do
		sum=sum+v
		n=n+1
	a=sum/n
	emit(1,a)
\end{lstlisting}

\subsection*{c)}
\begin{lstlisting}
function MAP(value)
	for all v in value do
		emit(v,1)

	
function REDUCE(key,values)
	emit(1,key)
\end{lstlisting}

\subsection*{d)}
\begin{lstlisting}
function MAP(value)
	for all v in value do
		emit(v,1)

	
function REDUCE(keys,values)
	cnt=0
	for all k in values do
		cnt=cnt+1
	emit(1,cnt)
		
\end{lstlisting}


\end{document}

